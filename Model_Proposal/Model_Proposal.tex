\documentclass[10pt]{beamer}

\usepackage[utf8]{inputenc}
\usepackage{beamerthemebars}

\usepackage{fontenc}
\usepackage{graphics}
\usepackage{hyperref}
\usepackage{multimedia}
\usepackage{hyperref}
\usepackage{tikz}
\usetikzlibrary{calc}
\usetikzlibrary{snakes}
\usepgflibrary{fpu}
\usetikzlibrary{cd}
\usetikzlibrary{angles} 
\usepackage{latexsym} % Símbolos                        ı
\usepackage{amsmath}
\usepackage{amssymb}
\usepackage{pifont}
\usepackage{animate}
\usepackage{fancybox}
\usepackage{media9}
\usepackage{color}
\usepackage{synttree}
%\newtheorem{acknowledgement}[theorem]{Acknowledgment}
%\newtheorem{algorithm}[theorem]{Algorithm}
\newtheorem{axiom}{Axiom}
\newtheorem{case}{Case}
\newtheorem{claim}{Claim}
\newtheorem{conclusion}{Conclusion}
\newtheorem{condition}{Condition}
\newtheorem{conjecture}{Conjecture}
\newtheorem{assumption}{Assumption}
\newtheorem{criterion}{Criterion}
\newtheorem{proposition}{Proposition}
\newtheorem{question}{Question}

\newcommand{\Ec}{\mathcal{E}}
\newcommand{\Ac}{\mathcal{A}}
\newcommand{\Vc}{\mathcal{V}}
\newcommand{\Dc}{\mathcal{D}}
\newcommand{\Sc}{\mathcal{S}}
\newcommand{\Nbb}{\mathbb{N}}
\newcommand{\Rbb}{\mathbb{R}}
\newcommand{\level}{\mathrm{level}}
\newcommand{\depth}{\mathrm{depth}}
\newcommand{\rank}{\mathrm{rank}}


\mode<presentation> {
  \usetheme{Warsaw}       % Tema seleccionado.
  \usecolortheme{crane}   % Color del tema.
  \setbeamercovered{transparent} % Transparencia.
}

\title{Model Proposal}
\author{Epidemiological Models -- COVID-19 Data Science Task Force}
\institute[]{} 
\titlegraphic{
%\includegraphics[width=1.5cm]{esi.png}~%
%   \includegraphics[width=3cm]{ceu_simbolo.png}~%
   }
\date{}


\begin{document}

\begin{frame}
\maketitle
\end{frame}

\begin{frame}
We consider $\Omega = \bigcup_{i \in K} \Omega_i$ a number of $\#K$-boxes and for
each unit period of time $[t_0,t_1]$ is divided in two regimes:
\begin{block}{First regime}
For $t_0 \le t \le t^*$ the system behaviour is a classical SEIR for
each $i \in K$ for a given $(\mu=0,\beta,\gamma,\sigma)$ and
$(S_i(t_0),E_i(t_0),I_i(t_0),R_i(t_0))$ we compute the solution
\begin{align}
\dot{S}_i(t) & = -\beta S_i \,I_i, \label{PSEIR1}\\
\dot{E}_i(t) & = \beta S_i \,I_i - \sigma E_i, \label{PSEIR2} \\
\dot{I}_i(t) & = \sigma E_i - \mu I_i, \label{PSEIR3}\\
\dot{R}_i(t) & = \gamma I_i - \mu R_i, \label{PSEIR4}
\end{align}
denoted by $(S_i(t),E_i(t),I_i(t),R_i(t))$ for $t_0 \le t \le t^*$
\end{block}

\end{frame}


\begin{frame}{Original model $\mu=0$}
\begin{block}{Second Regime I}
For $t^* \le t \le t_1$ and $i \in K$ let be
$S_{i,j}(t)$ (respectively, $I_{i,j}(t))$ denotes the susceptible individuals (the infected individuals) that lives in box $i$ and moves to 
box $j.$ In a similar way $N_{i,j}$ are the individuals that lives in box $i$ and moves to box $j.$
Clearly,
$$
N_i = \sum_{j \in K} N_{i,j}.
$$
Then ratio of infected individuals in the box $i$ is computed as
\begin{align*}
I_i(t) & = \frac{1}{N_i} \sum_{j \in K} I_{i,j}(t),
\end{align*}
where
$$
\sum_{j \in K} I_{i,j}(t) \sim \mathcal{B}\left( \sum_{j \in K} S_{i,j}(t^*), \beta
\frac{\sum_{j \in K} I_{i,j}(t^*)}{\sum_{j \in K}N_{i,j}}
 \right)
$$
\end{block}
\end{frame}


\begin{frame}{Original model $\mu=0$}
\begin{block}{Second Regime II}
Then ratio of susceptible individuals in the box $i$ is computed as
\begin{align*}
S_i(t) & = \frac{1}{N_i} \sum_{j \in K} S_{i,j}(t),
\end{align*}
where
$$
\sum_{j \in K} S_{i,j}(t) \sim \mathcal{B}\left( \sum_{j \in K} I_{i,j}(t^*), \beta
\frac{\sum_{j \in K} I_{i,j}(t^*)}{\sum_{j \in K}N_{i,j}}
 \right)
$$
\end{block}
\end{frame}

\end{document}

\begin{frame}{Probability Decomposition}
Take $A \in \{S,E,I,R\},$ by using the total probability formulae, we can decompose
\begin{align*}
A(t) &:= \mathbb{P}((t,\omega) \in \text{A}) \\
     & = \sum_{i \in I}  \mathbb{P}\left((t,\omega) \in \text{A}| (t,\omega) \in \{t\} \times \Omega_i\right)
     \mathbb{P}((t,\omega) \in \{t\} \times \Omega_i).
\end{align*}
Now,
$$
A_i(t):=  \mathbb{P}\left((t,\omega) \in \text{A}| (t,\omega) \in \{t\} \times \Omega_i\right)
$$
and
$$
P_i(t):= \mathbb{P}((t,\omega) \in \{t\} \times \Omega_i).
$$
Moreover, the dynamic of model changes as:
$$
\dot{A}(t) = \sum_{i \in I} \dot{A}_i(t)P_i(t) + A_i(t)\dot{P}_i(t).
$$
\end{frame}

\begin{frame}
It allows to the following system of equations:
\begin{align*}
\sum_{i \in I} \dot{S}_i(t)P_i(t) + S_i(t)\dot{P}_i(t) & = -\beta \left(\sum_{i \in I} S_i(t)P_i(t)\right)  \, \left(\sum_{i \in I} I_i(t)P_i(t)\right), \\
\sum_{i \in I} \dot{E}_i(t)P_i(t) + E_i(t)\dot{P}_i(t) & = \beta \left(\sum_{i \in I} S_i(t)P_i(t)\right) \,\left(\sum_{i \in I} I_i(t)P_i(t)\right) & \\
& - \sigma \left(\sum_{i \in I} E_i(t)P_i(t)\right),  \\
\sum_{i \in I} \dot{I}_i(t)P_i(t) + I_i(t)\dot{P}_i(t) & = \sigma \left(\sum_{i \in I} E_i(t)P_i(t)\right) - \mu \left(\sum_{i \in I} I_i(t)P_i(t)\right), \\
\sum_{i \in I} \dot{R}_i(t)P_i(t) + R_i(t)\dot{P}_i(t) & = \gamma \left(\sum_{i \in I} I_i(t)P_i(t)\right) - \mu \left(\sum_{i \in I} R_i(t)P_i(t)\right), 
\end{align*}
that we need to complete by including the population dynamics given by the bundle
$$
\{(P_i(t),\dot{P}_i(t)):i \in I\}.
$$
\end{frame}